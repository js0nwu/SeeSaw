%% Mathematical options & shortcuts

% A fix for parentess
\delimitershortfall=-1pt


% Required packages
\usepackage{ifthen}
\usepackage{mathtools} % Required for nice conditional expectation operator


% Widebar hack - due to 
\makeatletter
\newcommand*\rel@kern[1]{\kern#1\dimexpr\macc@kerna}
\newcommand*\widebar[1]{%
	\begingroup
	\def\mathaccent##1##2{%
		\rel@kern{0.8}%
		\overline{\rel@kern{-0.8}\macc@nucleus\rel@kern{0.2}}%
		\rel@kern{-0.2}%
	}%
	\macc@depth\@ne
	\let\math@bgroup\@empty \let\math@egroup\macc@set@skewchar
	\mathsurround\z@ \frozen@everymath{\mathgroup\macc@group\relax}%
	\macc@set@skewchar\relax
	\let\mathaccentV\macc@nested@a
	\macc@nested@a\relax111{#1}%
	\endgroup
}
\makeatother




% Fixing quotation - must use "etoolbox"
%\newtoggle{quotopen}
%\toggletrue{quotopen}
%%\catcode34=\active % lets you define `"` as a macro
%\DeclareRobustCommand{\testme}{%
%	\iftoggle{quotopen}{%
%		\togglefalse{quotopen} 
%		``
%	}{%
%	    \toggletrue{quotopen} ''
%	}
%}
% Deactive with: \catcode`\"=12\relax % changes `"` back to normal





% Equations referencing:
\newcommand{\Eq}[1]{Eq.~\eqref{eq:#1}\xspace}
\newcommand{\Eqs}[1]{Eqs.~\eqref{eq:#1}\xspace}

% Text related
\newcommand{\etal}{\textit{et al.}\xspace}
\newcommand{\apriori}{\textit{a priori}\xspace}
\newcommand{\aposteriori}{\textit{aposteriori}\xspace}
\newcommand{\eg}{\textit{e.g.}\xspace}
%\newcommand{\ie}{\textit{i.e.}\xspace}
\newcommand{\ie}{{i.e.}\xspace}
\newcommand{\etc}{\textit{etc}\xspace}

% Macros
\newcommand{\explain}[2]{\underset{\substack{\uparrow \\ \texttt{#2}}} {#1} } 


% Operators 
\newcommand{\dfn}{\triangleq}
\DeclareMathOperator{\prob}{Prob}
\DeclareMathOperator{\sign}{Sgn}


\DeclareMathOperator{\var}{Var}
\newcommand{\cov}{\mathsf{Cov}}

\DeclareMathOperator{\Jac}{J}
\newcommand{\Order}{\mathcal{O}}

\newcommand{\dOp}{\mathrm{d}}

% Refining forall operator
\let\oldforall\forall
\renewcommand{\forall}{\ensuremath{\; \oldforall \;}}




% Refining SUM operator
%\newcommand{\Sum}[0]{\sum\limits}
\newcommand{\Sum}[3]{\sum\limits_{#1}^{#2} {#3} \hspace{2pt}}

% Refining PROD operator
\newcommand{\Prod}[3]{\prod\limits_{#1}^{#2} {#3} \hspace{2pt}}

% Gradient (nabla) operator
\newcommand{\Grad}[2]{\nabla_{#1}{#2} \hspace{2pt}}

% Integral operator
\newcommand{\Int}[4]{\int\limits_{#1}^{#2}\hspace{-5pt} {#3} \hspace{2pt} \dOp{#4}\hspace{2pt}}


% A nicer transpose
%\DeclareMathOperator{\Transp}{T}
\newcommand{\TR}{{\hspace{-1pt}{\top}\hspace{-1pt}}}






% Symbols
%\newcommand{\diag}[1]{\mbox{diag}\bigl\{#1\bigr\}}
\newcommand{\diag}[1]{\mbox{diag}\left\{#1\right\}}
%\newcommand{\normal}[1]{\mathcal{N}\bigl(#1\bigr)}
\newcommand{\normal}[1]{\mathcal{N}(#1)}
\newcommand{\uniformal}[1]{\mathcal{U}\bigl[#1\bigr]}


%    Enclose the argument in vert-bar delimiters:
\newcommand{\envert}[1]{\left\lvert#1\right\rvert}
\let\abs=\envert
%
%    Enclose the argument in double-vert-bar delimiters:
\newcommand{\enVert}[1]{\left\lVert#1\right\rVert}
\let\norm=\enVert

% Norm of the vector/matrix
\newcommand{\Norm}[2][]{\enVert{#2}_{#1}}


% Vectors & Matrices
%\newcommand{\vc}[1]{}  %vector-matrix in BOLD
%\newcommand{\vc}[1]{#1}  %vector-matrix in BOLD
%\newcommand{\vc}[1]{\mathbold{#1}}  %vector-matrix in BOLD
%\renewcommand{\vec}[1]{\ensuremath{\boldsymbol{#1}}}
%\newcommand{\mtx}[1]{\ensuremath{\boldsymbol{#1}}}

% Prof. Tsiotras request - vectors & matrices are NOT BOLD
\renewcommand{\vec}[1]{\ensuremath{{#1}}}
\newcommand{\mtx}[1]{\ensuremath{{#1}}}



\newcommand{\func}[1]{#1}



% Braces
\newcommand{\br}[1]{{\hspace{-1pt}\left({#1}\right)}}
%\newcommand{\br}[1]{{\hspace{-1pt}(#1)}}
\newcommand{\brs}[1]{{\left[{#1}\right]}}
\newcommand{\brf}[1]{{\left\{{#1}\right\}}}

% Neglected order in linearizations:
\newcommand{\OfOrder}[1]{\Order\brf{#1}}

% Statistics:\
\newcommand{\Expectation}[1]{\mathbb{E}[#1]}
%\newcommand{\Expectation}{\E\expectarg}
%\DeclarePairedDelimiterX{\expectarg}[1]{[}{]}{%
%	\ifnum\currentgrouptype=16 \else\begingroup\fi
%	\activatebar#1
%	\ifnum\currentgrouptype=16 \else\endgroup\fi
%}
%
%\newcommand{\innermid}{\nonscript\;\delimsize\vert\nonscript\;}
%\newcommand{\activatebar}{%
%	\begingroup\lccode`\~=`\|
%	\lowercase{\endgroup\let~}\innermid 
%	\mathcode`|=\string"8000
%}


\newcommand{\ExpectationOver}[2]{\E_{#1}\brs{#2}}



\newcommand{\Cov}[1]{\cov\brs{#1}}
\newcommand{\Var}[2][]{\var_{#1}{#2}}                   % Central Moment 

% Fields
%\newcommand{\field}[1]{\:\mathbb{R}^{#1}}
%\newcommand{\Rfield}{\:\mathbb{R}}
%\newcommand{\Sfield}[1]{\:\mathbb{S}^{#1}}
\newcommand{\domain}[1]{\mathcal{#1}}
\newcommand{\Rfield}{\mathbb{R}}
\newcommand{\field}[1]{\Rfield^{#1}}
\newcommand{\Sfield}[1]{\:\mathbb{S}^{#1}}

\newcommand{\Cone}{\mathcal{C}^{1}}
\newcommand{\Ltwo}{\mathcal{L}_{2}}



\newcommand{\Reals}[1]{\field{#1}}
\newcommand{\Integers}[1]{\mathbb{Z}{#1}}

% Linear algebra
\DeclareMathOperator{\rank}{Rank}
\DeclareMathOperator{\nullity}{Nullity}

\newcommand{\rangespace}[1]{ \mathscr{R}\brf{#1} }
\newcommand{\nullspace}[1]{ \mathscr{N}\brf{#1} } 
\newcommand{\image}[1]{ \mathscr{I}\brf{#1}} 
\newcommand{\Rank}[1]{ \rank\brf{#1}}
\newcommand{\Nullity}[1]{ \nullity\brf{#1}}

\newcommand{\Jmat}[2]{\mtx{J}^{{#1},{#2}}}


% Sets
\newcommand{\given}{\ensuremath{ \ | \ }}




% Kronecker delta
\newcommand{\KronDelta}[1]{\delta_\br{#1}}



% Neighbouhood (ball)
\newcommand{\Ball}[2][]{{\mathrm{B}_{#1}}\br{#2}}
\newcommand{\nhd}{neighbourhood}





%%%%%%%%%%%%% ROTATIONS KINEMATICS %%%%%%%%%%%%%%%
% Cross Matrix 
\newcommand{\xMat}[1]{\brs{#1\times}}

% Euler Rotations
\newcommand{\rotX}[1]{\brs{#1}_x}
\newcommand{\rotY}[1]{\brs{#1}_y}
\newcommand{\rotZ}[1]{\brs{#1}_z}

% DCM from X to Y frames: usage \DCM{x}{y}
\newcommand{\DCM}[2]{\mtx{T}^{#1}_{#2}}
%%%%%%%%%%%%%%%%%%%%%%%%%%%%%%%%%%%%%%%%%%%%%%%%%%



%%%%%%%%%%%% General Math %%%%%%%%%%%%%%%
% Jacobian of vector1 to vector2 
%\newcommand{\jacvv}[3]{ \frac{\partial \vec{#1}}{\partial \vec{#2}} {\br{#3}}}
%\newcommand{\jacvv}[3]{\nabla_{\vec{#2}} \vec{#1}^T {\br{#3}}}
%\newcommand{\Jacobian}[3]{ \left. \brs{\nabla_{#2} {#1}^T}^T \right|_{#3}}

\newcommand{\Jacobian}[4][]{ \Jac^{#1}_{#3} \brf{{#2}\br{#4}} }




% Jacobian of var1 to var2
\newcommand{\jac}[3]{ \frac{\partial #1}{\partial #2} {\br{#3}}}

% Pardial derivative shortcut
\newcommand{\pder}[2]{ \frac{\partial{#1}}{\partial{#2}}}


% Full time derivative
\newcommand{\ddt}{\frac{\dOp}{\dOp t}}


% BMATRIX shortcut
\newcommand{\bmat}[1]{\begin{bmatrix} #1 \end{bmatrix}}


% Inverse
\newcommand{\inv}[1]{{#1}^{-1}}
\newcommand{\invb}[1]{\br{#1}^{\hspace{-2pt}-1}}



% Generic functions
\newcommand{\fun}[2]{#1\br{#2}}
\newcommand{\Dfun}[3]{{#1}_{#2}\br{#3}}
\newcommand{\DfunS}[4]{{#1}^{#2}_{#3}\br{#4}}


%%%%%%%%%%%%%%%%%%%%%%% Generic variables & vectors %%%%%%%%%%%%%%%%%%%%%%%



\newcommand{\funh}[0]{\fun{h}}
\newcommand{\fung}[0]{\fun{g}}

\newcommand{\funfb}[1]{\funf\br{#1}}
\newcommand{\funhb}[1]{\funh\br{#1}}
\newcommand{\fungb}[1]{\fung\br{#1}}



\newcommand{\I}[0]{\mathsf{I}}



%%%%%%%%%%%%%%%%% UNITS %%%%%%%%%%%%%%%%%%%%%
\newcommand{\rad}[0]{\,\texttt{rad}}
\newcommand{\mrad}[0]{\,\texttt{mrad}}
\newcommand{\radsec}[0]{\,\frac{\texttt{rad}}{\texttt{sec}}}
\newcommand{\radsecsec}[0]{\,\frac{\texttt{rad}}{\texttt{sec}^2}}
\newcommand{\mradsec}[0]{\,\frac{\texttt{mrad}}{\texttt{sec}}}

\newcommand{\meter}[0]{\,\texttt{m}}
\newcommand{\msec}[0]{\,\frac{\texttt{m}}{\texttt{sec}}}
\newcommand{\msecsec}[0]{\,\frac{\texttt{m}}{\texttt{sec}^2}}
%%%%%%%%%%%%%%%%%%%%%%%%%%%%%%%%%%%%%%%%%%%%%



%%%%%%%%%%%%% Vector spaces %%%%%%%%%%%%%%
\newcommand{\vecspace}[1]{\mathcal{#1}}


% % % % % % % % % % % % % % % % Shortcuts % % % % % % % %
\newcommand{\half}{\frac{1}{2}}


%% Matrix calculus
\DeclareMathOperator{\Tr}{Tr}
\DeclareMathOperator{\VecOp}{vec}
% Symbols & function
\newcommand{\Det}[1]{|#1|} % Determinant
\newcommand{\Trace}[1]{\Tr\brs{#1}} % Trace
\newcommand{\VecFun}[1]{\VecOp\br{#1}}


% Commutation matrix for some matrix
\newcommand{\CommutMat}[1]{\mtx{T}_{#1}}
\newcommand{\CommutInvMat}[1]{\widetilde{\mtx{T}}_{#1}}

% Kronecker product
\newcommand{\Kron}[2]{\br{{#1} \otimes {#2}}}


% Arcmin






\endinput
%%% Local Variables: 
%%% mode: latex
%%% TeX-master: "main"
%%% End: 
